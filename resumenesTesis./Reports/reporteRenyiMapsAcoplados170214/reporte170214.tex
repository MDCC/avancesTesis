
%%\documentclass[preprint,12pt,3p]{elsarticle}
\documentclass[12pt,3p]{elsarticle}

\usepackage{amssymb}

\usepackage[utf8]{inputenc}
\usepackage[spanish]{babel}
\usepackage{graphicx}
\usepackage{amsmath}
\usepackage{subfigure} 
\usepackage{float}
% For algorithms
\usepackage{algorithm}
\usepackage{algorithmic}





\journal{--}

\begin{document}

\begin{frontmatter}

\title{Implementación de Renyi Maps acoplados y aplicación de pruebas NIST.}



\author{Marcos Daniel Calderón Calderón}



\ead{marcos.calderon@cimat.mx}




\begin{abstract}
En este reporte, se explica a detalle la implementación de varios mapas caóticos, además, se hace un análisis del valor acoplado H que se genera en cada algoritmo.
\end{abstract}



\end{frontmatter}

\section{Introducción.}

A continuación, se especifican algunos criterios para la implementación de mapas caóticos acoplados. 

\subsection{Criterio 1.}

El primer criterio utilizado es el siguiente:

\begin{equation}
X_{i,j}= f(X_{i,j-1})+ \epsilon  H(X_{i,j-1},...,X_{N,j-1})
\end{equation}

donde:
\begin{equation}
H(X_{i,j-1},...,X_{N,j-1}) = \sum_{i=1}^{N}(X_{i,j-1}\mod 256).
\end{equation}
además, $\epsilon$ es un valor aleatorio del conjunto: ${\{-1, 0, 1 \}}$




\subsection{Criterio 2.}

El segundo criterio utilizado es el siguiente:

\begin{equation}
X_{i,j}= f(X_{i,j-1})+ \epsilon  H(X_{i,j-1},...,X_{N,j-1})
\end{equation}

donde:
\begin{equation}
H(X_{i,j-1},...,X_{N,j-1}) = \bigoplus _{i=1}^{N}(X_{i,j-1}).
\end{equation}
donde $\bigoplus$ representa la operación $XOR$, además $\epsilon$ es un valor aleatorio del conjunto: ${\{-1, 0, 1 \}}$



\subsection{Criterio 3.}

El tercer criterio utilizado es el siguiente:

\begin{equation}
X_{i,j}= \hat{\gamma} \bigoplus f(X_{i,j-1})+ \gamma \bigoplus H(X_{i,j-1},...,X_{N,j-1})
\end{equation}

donde:
\begin{equation}
H(X_{i,j-1},...,X_{N,j-1}) = \bigoplus _{i=1}^{N}(X_{i,j-1}).
\end{equation}
donde $\bigoplus$ representa la operación $XOR$, además para el cálculo del valor de $\gamma$, necesitamos un $\epsilon$ que será un valor aleatorio donde $1 \geq \epsilon \geq 32$, el rango es establecido de acuerdo al tipo de dato utilizado cuando se traduce el sistema a un lenguaje de programación elegido. Ahora, podemos calcular $\gamma$ de la siguiente manera:

\begin{equation}
\gamma = 2^{\epsilon}-1 \quad  
\end{equation}

\begin{equation}
\hat{\gamma} = M - \gamma
\end{equation}

donde $M = 2^{32}-1$.


\section{Detalles de implementación.}
A continuación, se especifican algunos detalles importantes a tomar en cuenta a la hora de la implementación de los mapas acoplados:

\begin{itemize}

\En el criterio 1 y 2, es importante obtener valores de H que no sean muy grandes, esto se busca con la finalidad de no hacer perturbaciones muy grandes. Tambi

\item Se buscaron 80000 valores en la ejecución del programa; además, cada valor está compuesto de 32 bits cuando se guarda la información en un archivo binario. Por las características mencionadas anteriormente, se pueden leer \textbf{2,560,000 bits} para la aplicación de las pruebas NIST.

\item Para equipos de cómputo de 32 bits, se utilizó el tipo de dato \textbf{unsigned long}, que está conformado de 4 bytes. Si el equipo de cómputo utilizado es de 64 bits, se recomienta utilizar el tipo de dato \textbf{unsigned int}, todavía no hay un estándar definido para el tamaño de los tipos de datos en los equipos de 64 bits.
\end{itemize}





\section{Resultados.}

\subsection{Aplicación de las pruebas NIST a los mapas acoplados del Criterio 1.}
Recordemos que el criterio 1 era el siguiente:

\begin{equation}
X_{i,j}= f(X_{i,j-1})+ \epsilon  H(X_{i,j-1},...,X_{N,j-1})
\end{equation}

donde:
\begin{equation}
H(X_{i,j-1},...,X_{N,j-1}) = \sum_{i=1}^{N}(X_{i,j-1}\mod 256).
\end{equation}
además, $\epsilon$ es un valor aleatorio del conjunto: ${\{-1, 0, 1 \}}$



Se ejecutó el siguiente código:

\begin{itemize}
\item \textbf{./assess 2560000}
\item User Prescribed Input File: \textbf{binarioSUMA.dat}
\item    Enter 0 if you DO NOT want to apply all of the
         statistical tests to each sequence and 1 if you DO. Enter chice: \textbf{1}
                  
\item  How many bitstreams? \textbf{1}

\item Input File Format:
    [0] ASCII - A sequence of ASCII 0's and 1's
    [1] Binary - Each byte in data file contains 8 bits of data

   Select input mode:  \textbf{1}
\end{itemize}


Se obtuvieron los siguientes resultados:

\begin{table}[!h]
\caption{Resultados de las pruebas de aleatoriedad NIST a los datos binarioSUMA.dat .}
\label{sample-table}
\vskip 0.15in
\begin{center}
\begin{small}
\begin{sc}
\begin{tabular}{lccr}
\hline

Prueba Aplicada &  P-Valor & Exito? \\
\hline

Aproximate Entropy    &   0.241029  & $\surd$ \\

Block Frecuency  &  0.464753 &  $\surd$  \\

Cumulative Sums    &   Forward test:  0.628651, Reverse test: 0.848525  & $\surd$ \\

FFT    &   0.149985 &   $\surd$      \\

Frecuency     &  0.625019 &  $\surd$   \\

Linear Complexity      &  0.595287  & $\surd$ \\

Longest Run      &   0.511949  &    $\surd$      \\

Non Overlapping Template      & P-valores aceptados: 145 de 148    &     $\surd$          \\

Overlapping Template      &  0.037821  &        $\surd$       \\

Random Excursions      & P-valores aceptados: 8 de 8  &     $\surd$          \\

Random Excursions Variant & P-valores aceptados: 18 de 18  &    $\surd$        \\

Rank &    0.111388    &       $\surd$      \\

Runs &        0.957015  &     $\surd$        \\

Serial &     P-valores aceptados: 2 de 2    &     $\surd$        \\

Universal &     0.136601 &   $\surd$            \\

\hline



\end{tabular}
\end{sc}
\end{small}
\end{center}
\vskip -0.1in
\end{table}


\subsection{Aplicación de las pruebas NIST a los mapas acoplados del Criterio 2.}

El criterio 2 era el siguiente:
\begin{equation}
X_{i,j}= f(X_{i,j-1})+ \epsilon  H(X_{i,j-1},...,X_{N,j-1})
\end{equation}

donde:
\begin{equation}
H(X_{i,j-1},...,X_{N,j-1}) = \bigoplus _{i=1}^{N}(X_{i,j-1}).
\end{equation}
donde $\bigoplus$ representa la operación $XOR$, además $\epsilon$ es un valor aleatorio del conjunto: ${\{-1, 0, 1 \}}$


Para el criterio 2, se ejecutó el siguiente código:

\begin{itemize}
\item \textbf{./assess 2560000}
\item User Prescribed Input File: \textbf{binarioXOR.dat}
\item    Enter 0 if you DO NOT want to apply all of the
         statistical tests to each sequence and 1 if you DO. Enter chice: \textbf{1}
                  
\item  How many bitstreams? \textbf{1}

\item Input File Format:
    [0] ASCII - A sequence of ASCII 0's and 1's
    [1] Binary - Each byte in data file contains 8 bits of data

   Select input mode:  \textbf{1}
\end{itemize}


Se obtuvieron los siguientes resultados:

\begin{table}[!h]
\caption{Resultados de las pruebas de aleatoriedad NIST a los datos binarioXOR.dat .}
\label{sample-table}
\vskip 0.15in
\begin{center}
\begin{small}
\begin{sc}
\begin{tabular}{lccr}
\hline

Prueba Aplicada &  P-Valor & Exito? \\
\hline

Aproximate Entropy    &   0.858228  & $\surd$ \\

Block Frecuency  & 0.118270  &  $\surd$  \\

Cumulative Sums    &   Forward test: 0.021856, Reverse test: 0.035757  & $\surd$ \\

FFT    &   0.787496 &   $\surd$      \\

Frecuency     &  0.022608 &  $\surd$   \\

Linear Complexity      &  0.791809  & $\surd$ \\

Longest Run      &   0.378018  &    $\surd$      \\

Non Overlapping Template      & P-valores aceptados: 144 de 148    &     $\surd$          \\

Overlapping Template      &  0.457920 &        $\surd$       \\

Random Excursions      &  NOT APPLICABLE. &             \\

Random Excursions Variant & NOT APPLICABLE. &          \\

Rank &    0.278871   &       $\surd$      \\

Runs &    0.019569  &     $\surd$        \\

Serial &     P-valores aceptados: 2 de 2    &     $\surd$        \\

Universal &      0.249147 &   $\surd$            \\

\hline



\end{tabular}
\end{sc}
\end{small}
\end{center}
\vskip -0.1in
\end{table}


\subsection{Aplicación de las pruebas NIST a los mapas acoplados del Criterio 3.}

EL criterio 3 era el siguiente:
\begin{equation}
X_{i,j}= \hat{\gamma} \bigoplus f(X_{i,j-1})+ \gamma \bigoplus H(X_{i,j-1},...,X_{N,j-1})
\end{equation}

donde:
\begin{equation}
H(X_{i,j-1},...,X_{N,j-1}) = \bigoplus _{i=1}^{N}(X_{i,j-1}).
\end{equation}



Para el criterio 3, se ejecutó el siguiente código:

\begin{itemize}
\item \textbf{./assess 2560000}
\item User Prescribed Input File: \textbf{binarioXORcomp.dat}
\item    Enter 0 if you DO NOT want to apply all of the
         statistical tests to each sequence and 1 if you DO. Enter chice: \textbf{1}
                  
\item  How many bitstreams? \textbf{1}

\item Input File Format:
    [0] ASCII - A sequence of ASCII 0's and 1's
    [1] Binary - Each byte in data file contains 8 bits of data

   Select input mode:  \textbf{1}
\end{itemize}


Se obtuvieron los siguientes resultados:

\begin{table}[!h]
\caption{Resultados de las pruebas de aleatoriedad NIST a los datos binarioXORcomp.dat .}
\label{sample-table}
\vskip 0.15in
\begin{center}
\begin{small}
\begin{sc}
\begin{tabular}{lccr}
\hline

Prueba Aplicada &  P-Valor & Exito? \\
\hline

Aproximate Entropy    &    0.979174  & $\surd$ \\

Block Frecuency  &  0.283892 &  $\surd$  \\

Cumulative Sums    &   Forward test:  0.320736, Reverse test: 0.476676  & $\surd$ \\

FFT    &   0.990848 &   $\surd$      \\

Frecuency     &  0.286876 &  $\surd$   \\

Linear Complexity      &  0.142002 & $\surd$ \\

Longest Run      &    0.106050 &    $\surd$      \\

Non Overlapping Template      & P-valores aceptados: 145 de 148    &     $\surd$          \\

Overlapping Template      &  0.879647  &        $\surd$       \\

Random Excursions      & P-valores aceptados: 8 de 8  &     $\surd$          \\

Random Excursions Variant & P-valores aceptados: 18 de 18  &    $\surd$        \\

Rank &    0.398102    &       $\surd$      \\

Runs &        0.964539  &     $\surd$        \\

Serial &     P-valores aceptados: 2 de 2    &     $\surd$        \\

Universal &     0.603939  &   $\surd$            \\

\hline



\end{tabular}
\end{sc}
\end{small}
\end{center}
\vskip -0.1in
\end{table}


\section{Conclusiones.}
En general, los criterios 1 y 3 arrojan mejores resultados que el criterio 2. Si se tuviera que elegir entre el criterio 1 y el 3. Se puede concluir que el criterio 3 obtuvo un mejor desempeño de acuerdo a los p-valores obtenidos.

\section{Anexos.}


\end{document}

