
%%\documentclass[preprint,12pt,3p]{elsarticle}
\documentclass[12pt,3p]{elsarticle}

\usepackage{amssymb}

\usepackage[utf8]{inputenc}
\usepackage[spanish]{babel}
\usepackage{graphicx}
\usepackage{amsmath}
\usepackage{subfigure} 
\usepackage{float}
% For algorithms
\usepackage{algorithm}
\usepackage{algorithmic}





\journal{--}

\begin{document}

\begin{frontmatter}

\title{Implementación de Renyi Maps acoplados (modificación) y aplicación de pruebas NIST.}



\author{Marcos Daniel Calderón Calderón}



\ead{marcos.calderon@cimat.mx}




\begin{abstract}
En este reporte, se explica a detalle la implementación de varios mapas caóticos, además, se hace un análisis del valor acoplado H que se genera en cada algoritmo.
\end{abstract}



\end{frontmatter}

\section{Introducción.}

A continuación, se especifican algunos criterios para la implementación de mapas caóticos acoplados. 

\subsection{Criterio 1.}

El primer criterio utilizado es el siguiente:

\begin{equation}
X_{i,j}= f(X_{i,j-1})+ \epsilon  H(X_{i,j-1},...,X_{N,j-1})
\end{equation}

donde:
\begin{equation}
H(X_{i,j-1},...,X_{N,j-1}) = \sum_{i=1}^{N}(X_{i,j-1}\mod 256).
\end{equation}
además, $\epsilon$ es un valor aleatorio del conjunto: ${\{-1, 0, 1 \}}$




\subsection{Criterio 2.}

El segundo criterio utilizado es el siguiente:

\begin{equation}
X_{i,j}= f(X_{i,j-1})+ \epsilon  H(X_{i,j-1},...,X_{N,j-1})
\end{equation}

donde:
\begin{equation}
H(X_{i,j-1},...,X_{N,j-1}) = \bigoplus _{i=1}^{N}(X_{i,j-1}).
\end{equation}
donde $\bigoplus$ representa la operación $XOR$, además $\epsilon$ es un valor aleatorio del conjunto: ${\{-1, 0, 1 \}}$



\subsection{Criterio 3.}

El tercer criterio utilizado es el siguiente:

\begin{equation}
X_{i,j}= \hat{\gamma} \bigoplus f(X_{i,j-1})+ \gamma \bigoplus H(X_{i,j-1},...,X_{N,j-1})
\end{equation}

donde:
\begin{equation}
H(X_{i,j-1},...,X_{N,j-1}) = \bigoplus _{i=1}^{N}(X_{i,j-1}).
\end{equation}
donde $\bigoplus$ representa la operación $XOR$, además para el cálculo del valor de $\gamma$, necesitamos un $\epsilon$ que será un valor aleatorio donde $1 \geq \epsilon \geq 32$, el rango es establecido de acuerdo al tipo de dato utilizado cuando se traduce el sistema a un lenguaje de programación elegido. Ahora, podemos calcular $\gamma$ de la siguiente manera:

\begin{equation}
\gamma = 2^{\epsilon}-1 \quad  
\end{equation}

\begin{equation}
\hat{\gamma} = M - \gamma
\end{equation}

donde $M = 2^{32}-1$.


\section{Detalles de implementación.}
A continuación, se especifican algunos detalles importantes a tomar en cuenta a la hora de la implementación de los mapas acoplados:

\begin{itemize}

\item En el criterio 1 y 2, es importante obtener valores de H que no sean muy grandes, esto se busca con la finalidad de no hacer perturbaciones muy grandes. Es necesario analizar los valores de H que son generados. 

\item Se buscaron 80000 valores en la ejecución del programa; además, cada valor está compuesto de 32 bits cuando se guarda la información en un archivo binario. Por las características mencionadas anteriormente, se pueden leer \textbf{2,560,000 bits} para la aplicación de las pruebas NIST.

\item Para equipos de cómputo de 32 bits, se utilizó el tipo de dato \textbf{unsigned long}, que está conformado de 4 bytes. Si el equipo de cómputo utilizado es de 64 bits, se recomienta utilizar el tipo de dato \textbf{unsigned int}, todavía no hay un estándar definido para el tamaño de los tipos de datos en los equipos de 64 bits.
\end{itemize}


\section{Resultados.}

\subsection{Aplicación de las pruebas NIST a los mapas acoplados del Criterio 1.}
Recordemos que el criterio 1 era el siguiente:

\begin{equation}
X_{i,j}= f(X_{i,j-1})+ \epsilon  H(X_{i,j-1},...,X_{N,j-1})
\end{equation}

donde:
\begin{equation}
H(X_{i,j-1},...,X_{N,j-1}) = \sum_{i=1}^{N}(X_{i,j-1}\mod 256).
\end{equation}
además, $\epsilon$ es un valor aleatorio del conjunto: ${\{-1,  1 \}}$


En este ejemplo, se hizo una modificación en la forma en cómo se obtuvo H. Se ejecutó el siguiente código:

\begin{itemize}
\item \textbf{./assess 2560000}
\item User Prescribed Input File: \textbf{binarioSUMAmod.dat}
\item    Enter 0 if you DO NOT want to apply all of the
         statistical tests to each sequence and 1 if you DO. Enter chice: \textbf{1}
                  
\item  How many bitstreams? \textbf{1}

\item Input File Format:
    [0] ASCII - A sequence of ASCII 0's and 1's
    [1] Binary - Each byte in data file contains 8 bits of data

   Select input mode:  \textbf{1}
\end{itemize}


Se obtuvieron los siguientes resultados:

\begin{table}[H]
\caption{Resultados de las pruebas de aleatoriedad NIST a los datos binarioSUMAmod.dat .}
\label{sample-table}
\vskip 0.15in
\begin{center}
\begin{small}
\begin{sc}
\begin{tabular}{lccr}
\hline

Prueba Aplicada &  P-Valor & Exito? \\
\hline

Aproximate Entropy    &   0.561487  & $\surd$ \\

Block Frecuency  &  0.123096 &  $\surd$  \\

Cumulative Sums    &   Forward test:  0.393884, Reverse test: 0.215339  & $\surd$ \\

FFT    &  0.235138 &   $\surd$      \\

Frecuency     &  0.216366 &  $\surd$   \\

Linear Complexity      & 0.639128  & $\surd$ \\

Longest Run      &   0.680758  &    $\surd$      \\

Non Overlapping Template      & P-valores aceptados: 148 de 148    &     $\surd$          \\

Overlapping Template      &  0.679278  &        $\surd$       \\

Random Excursions      & P-valores aceptados: 8 de 8  &     $\surd$          \\

Random Excursions Variant & P-valores aceptados: 18 de 18  &    $\surd$        \\

Rank &   0.286919   &       $\surd$      \\

Runs &      0.091564 &     $\surd$        \\

Serial &     P-valores aceptados: 2 de 2    &     $\surd$        \\

Universal &     0.990027 &   $\surd$            \\

\hline



\end{tabular}
\end{sc}
\end{small}
\end{center}
\vskip -0.1in
\end{table}


\subsection{Aplicación de las pruebas NIST a los mapas acoplados del Criterio 2.}

El criterio 2 era el siguiente:
\begin{equation}
X_{i,j}= f(X_{i,j-1})+ \epsilon  H(X_{i,j-1},...,X_{N,j-1})
\end{equation}

donde:
\begin{equation}
H(X_{i,j-1},...,X_{N,j-1}) = \bigoplus _{i=1}^{N}(X_{i,j-1}).
\end{equation}
donde $\bigoplus$ representa la operación $XOR$, además $\epsilon$ es un valor aleatorio del conjunto: ${\{-1, 1 \}}$

En este ejemplo, se hizo una modificación en la forma en cómo se obtuvo H. Se ejecutó el siguiente código:
Para el criterio 2, se ejecutó el siguiente código:

\begin{itemize}
\item \textbf{./assess 2560000}
\item User Prescribed Input File: \textbf{binarioXORmod.dat}
\item    Enter 0 if you DO NOT want to apply all of the
         statistical tests to each sequence and 1 if you DO. Enter chice: \textbf{1}
                  
\item  How many bitstreams? \textbf{1}

\item Input File Format:
    [0] ASCII - A sequence of ASCII 0's and 1's
    [1] Binary - Each byte in data file contains 8 bits of data

   Select input mode:  \textbf{1}
\end{itemize}


Se obtuvieron los siguientes resultados:

\begin{table}[H]
\caption{Resultados de las pruebas de aleatoriedad NIST a los datos binarioXORmod.dat .}
\label{sample-table}
\vskip 0.15in
\begin{center}
\begin{small}
\begin{sc}
\begin{tabular}{lccr}
\hline

Prueba Aplicada &  P-Valor & Exito? \\
\hline

Aproximate Entropy    &   0.955023  & $\surd$ \\

Block Frecuency  & 0.932697  &  $\surd$  \\

Cumulative Sums    &   Forward test:0.532788, Reverse test: 0.466893  & $\surd$ \\

FFT    &   0.349856 &   $\surd$      \\

Frecuency     &  0.935243 &  $\surd$   \\

Linear Complexity      &  0.014630  & $\surd$ \\

Longest Run      &  0.014097 &    $\surd$      \\

Non Overlapping Template      & P-valores aceptados: 145 de 148    &     $\surd$          \\

Overlapping Template      &  0.129367  &        $\surd$       \\

Random Excursions      &  P-valores aceptados: 8 de 8  &      $\surd$          \\

Random Excursions Variant &  P-valores aceptados: 14 de 18  &     $\surd$        \\

Rank &    0.001981  &       $ No. $      \\

Runs &    0.865007  &     $\surd$        \\

Serial &     P-valores aceptados: 2 de 2    &     $\surd$        \\

Universal &      0.425674 &   $\surd$            \\

\hline



\end{tabular}
\end{sc}
\end{small}
\end{center}
\vskip -0.1in
\end{table}


\subsection{Aplicación de las pruebas NIST a los mapas acoplados del Criterio 3.}

EL criterio 3 era el siguiente:
\begin{equation}
X_{i,j}= \hat{\gamma} \bigoplus f(X_{i,j-1})+ \gamma \bigoplus H(X_{i,j-1},...,X_{N,j-1})
\end{equation}

donde:
\begin{equation}
H(X_{i,j-1},...,X_{N,j-1}) = \bigoplus _{i=1}^{N}(X_{i,j-1}).
\end{equation}



Para el criterio 3, se ejecutó el siguiente código:

\begin{itemize}
\item \textbf{./assess 2560000}
\item User Prescribed Input File: \textbf{binarioXORcomp.dat}
\item    Enter 0 if you DO NOT want to apply all of the
         statistical tests to each sequence and 1 if you DO. Enter chice: \textbf{1}
                  
\item  How many bitstreams? \textbf{1}

\item Input File Format:
    [0] ASCII - A sequence of ASCII 0's and 1's
    [1] Binary - Each byte in data file contains 8 bits of data

   Select input mode:  \textbf{1}
\end{itemize}


Se obtuvieron los siguientes resultados:

\begin{table}[H]
\caption{Resultados de las pruebas de aleatoriedad NIST a los datos binarioXORcomp.dat .}
\label{sample-table}
\vskip 0.15in
\begin{center}
\begin{small}
\begin{sc}
\begin{tabular}{lccr}
\hline

Prueba Aplicada &  P-Valor & Exito? \\
\hline

Aproximate Entropy    &    0.979174  & $\surd$ \\

Block Frecuency  &  0.283892 &  $\surd$  \\

Cumulative Sums    &   Forward test:  0.320736, Reverse test: 0.476676  & $\surd$ \\

FFT    &   0.990848 &   $\surd$      \\

Frecuency     &  0.286876 &  $\surd$   \\

Linear Complexity      &  0.142002 & $\surd$ \\

Longest Run      &    0.106050 &    $\surd$      \\

Non Overlapping Template      & P-valores aceptados: 145 de 148    &     $\surd$          \\

Overlapping Template      &  0.879647  &        $\surd$       \\

Random Excursions      & P-valores aceptados: 8 de 8  &     $\surd$          \\

Random Excursions Variant & P-valores aceptados: 18 de 18  &    $\surd$        \\

Rank &    0.398102    &       $\surd$      \\

Runs &        0.964539  &     $\surd$        \\

Serial &     P-valores aceptados: 2 de 2    &     $\surd$        \\

Universal &     0.603939  &   $\surd$            \\

\hline



\end{tabular}
\end{sc}
\end{small}
\end{center}
\vskip -0.1in
\end{table}

\section{Resultados para H.}

En este ejemplo, se tomaron 20,000 valores generados de H en cada uno de los criterios, como cada uno de estos valores está representado en 32 bits, el archivo binario estuvo compuesto de 640,000 bits.

\subsection{Analisis de los valores H generados para el criterio 1.}

\begin{table}[H]
\caption{Resultados de las pruebas de aleatoriedad NIST a los datos HdeSuma.dat .}
\label{sample-table}
\vskip 0.15in
\begin{center}
\begin{small}
\begin{sc}
\begin{tabular}{lccr}
\hline

Prueba Aplicada &  P-Valor & Exito? \\
\hline

Aproximate Entropy    &   0.935134 & $\surd$ \\

Block Frecuency  & 0.122908  &  $\surd$  \\

Cumulative Sums    &   Forward test:0.738816, Reverse test: 0.596414  & $\surd$ \\

FFT    &   0.818546 &   $\surd$      \\

Frecuency     &  0.439818 &  $\surd$   \\

Linear Complexity      &  0.813844  & $\surd$ \\

Longest Run      &  0.108021 &    $\surd$      \\

Non Overlapping Template      & P-valores aceptados: 147 de 148    &     $\surd$          \\

Overlapping Template      &  0.058303  &        $\surd$       \\

Random Excursions      &  P-valores aceptados: 8 de 8  &      $\surd$          \\

Random Excursions Variant &  P-valores aceptados: 18 de 18  &     $\surd$        \\

Rank &    0.847382   &        $\surd$         \\

Runs &    0.697088  &     $\surd$        \\

Serial &     P-valores aceptados: 2 de 2    &     $\surd$        \\

Universal &      0.673427 &   $\surd$            \\

\hline



\end{tabular}
\end{sc}
\end{small}
\end{center}
\vskip -0.1in
\end{table}



\subsection{Analisis de los valores H generados para el criterio 2.}


\begin{table}[H]
\caption{Resultados de las pruebas de aleatoriedad NIST a los datos HdeXOR.dat .}
\label{sample-table}
\vskip 0.15in
\begin{center}
\begin{small}
\begin{sc}
\begin{tabular}{lccr}
\hline

Prueba Aplicada &  P-Valor & Exito? \\
\hline

Aproximate Entropy    &   0.197412  & $\surd$ \\

Block Frecuency  & 0.009954 &  No. \\

Cumulative Sums    &   Forward test:0.105981, Reverse test: 0.148928  & $\surd$ \\

FFT    &   0.543220 &   $\surd$      \\

Frecuency     &  0.075898 &  $\surd$   \\

Linear Complexity      &  0.863083  & $\surd$ \\

Longest Run      &  0.072816 &    $\surd$      \\

Non Overlapping Template      & P-valores aceptados: 147 de 148    &     $\surd$          \\

Overlapping Template      &  0.173029  &        $\surd$       \\

Random Excursions      &  NO APLICABLE &       \\

Random Excursions Variant &  NO APLICABLE  &      \\

Rank &    0.217328   &       $\surd$       \\

Runs &    0.953301 &     $\surd$        \\

Serial &     P-valores aceptados: 2 de 2    &     $\surd$        \\

Universal &      0.993846 &   $\surd$            \\

\hline



\end{tabular}
\end{sc}
\end{small}
\end{center}
\vskip -0.1in
\end{table}


\subsection{Analisis de los valores H generados para el criterio 3.}


\begin{table}[H]
\caption{Resultados de las pruebas de aleatoriedad NIST a los datos HdeXORcomp.dat .}
\label{sample-table}
\vskip 0.15in
\begin{center}
\begin{small}
\begin{sc}
\begin{tabular}{lccr}
\hline

Prueba Aplicada &  P-Valor & Exito? \\
\hline

Aproximate Entropy    &   0.161036  & $\surd$ \\

Block Frecuency  & 0.944077  &  $\surd$  \\

Cumulative Sums    &   Forward test:0.294090, Reverse test: 0.645302  & $\surd$ \\

FFT    &   0.415401 &   $\surd$      \\

Frecuency     &  0.640142 &  $\surd$   \\

Linear Complexity      &  0.527570  & $\surd$ \\

Longest Run      &  0.948689 &    $\surd$      \\

Non Overlapping Template      & P-valores aceptados: 147 de 148    &     $\surd$          \\

Overlapping Template      &  0.371232 &        $\surd$       \\

Random Excursions      &  NO APLICABLE &      \\

Random Excursions Variant &  NO APLICABLE  &      \\

Rank &    0.301543  &      $\surd$      \\

Runs &    0.787370  &     $\surd$        \\

Serial &     P-valores aceptados: 2 de 2    &     $\surd$        \\

Universal &      0.896941 &   $\surd$            \\

\hline



\end{tabular}
\end{sc}
\end{small}
\end{center}
\vskip -0.1in
\end{table}



\section{Conclusiones.}
En general, los criterios 1 y 3 arrojan mejores resultados que el criterio 2. Si se tuviera que elegir entre el criterio 1 y el 3. Se puede concluir que el criterio 3 obtuvo un mejor desempeño de acuerdo a los p-valores obtenidos.

Para el caso de los valores de H, otra vez, se obtuvieron mejores resultados en el criterio 1 y en el criterio 3.
Sin embargo, las pruebas ''Random Excursions'' y ''Random Excursions Variant'' no siempre se pudieron aplicar.
\section{Anexos.}

\subsection{Código del criterio 1.}
\begin{verbatim}
#include <stdio.h>
#include <stdlib.h>
#include <math.h>


#define RENYI_MAP(var, parametro, j) ((var)*(parametro)+((var)>>(j)))

#define  MAX 4294967295 /*El valor de 2^32-1.*/
#define TAMANIOCICLO 4294967296 /*EL valor de 2^32.*/
#define noMapas 4 /*NUmero de mapas a utilizar.*/
#define ITtotales 80000 /*Iteraciones totales para NIST.*/
#define tamanioH 20000 /*Tamanio del arreglo de H.*/


/*
 File:   main.c
 Author: daniel
 El siguiente programa es un ejemplo de cuatro mapas caoticos RENYI acoplados.
 Se utiliza la suma para lograr el acoplado.
 Para declarar nuestras variables, utilizamos los siguientes componentes:
 -Un arreglo que guarda el valor de los parametros para cada mapa.
 -Un arreglo que guarda el valor de los valores calculados para cada mapa.
*/


int main(){
 
   
   /*Declaramos los arreglos que vamos a utilizar para guardar esto.*/
   unsigned long Xn[noMapas];
   unsigned long Xtotal[ITtotales]; 
   unsigned long parametros[noMapas];
   unsigned int i;
   unsigned long arregloH[tamanioH];
   int epsilon;
   FILE *  archivobin; 
   FILE *  archivoH;
   

   unsigned long H=0; 
   unsigned int j=9; /*No mas de 16.*/
   unsigned long iteraciones=0;
   unsigned long IT = 80000;


  /* Apertura del fichero de destino, para escritura en binario.*/
   archivobin = fopen ("binarioSUMAmod.dat", "wb");
   if (archivobin==NULL)
   {
   perror("No se puede abrir binarioSUMAmod.dat");
   return -1;
   }
   
   /* Apertura del archivo H para su escritura en binario.*/
   archivoH = fopen ("HdeSuma.dat", "wb");
   if (archivoH==NULL)
   {
   perror("No se puede abrir HdeSuma.dat");
   return -1;
   }
   
   
   


   /*Inicializamos nuestros parametros, en este punto se aplica el uso
   de una llave, en este ejemplo todavia no elegimos una. Tambien,
   los parametros son fijos en este ejemplo.*/
   parametros[0]=131071;
   Xn[0]=653;
   parametros[1]=104729;
   Xn[1]=769;
   parametros[2]=524287;
   Xn[2]=227;
   parametros[3]=65537;
   Xn[3]=823;
                  
   int contDeH=0;
   
   /*Primero, hacemos un ciclo inicial para calcular un nuevo valor para cada
   uno de los mapas y calcular, por primera vez, el resultado de la operacion
   XOR.*/
   for( i =0; i<noMapas; i++){
       Xn[i]= RENYI_MAP(Xn[i],parametros[i],j);
       H+=Xn[i];
   }
    
    arregloH[contDeH]=H; 
    contDeH++;
   
   /*Tambien al primer H se le aplica el mod.*/
   H%=256; 
   
   
   /*Elegimos un epsilon aleatorio entre 1 y 16 (para operaciones de 32 bits).
   En este caso, elegimos uno que este entre 0 y 15.*/
   epsilon = 1;   
  
      
   unsigned int k;
   unsigned long newH;
   
  
   do {
        
        newH = 0;
        for(k=0;k<noMapas; k++){            
            Xn[k]= RENYI_MAP(Xn[k],parametros[k],j) + epsilon*H;
            Xtotal[iteraciones++] = Xn[k];
             
            newH+=(Xn[k]);      
        }
        H = newH;
        arregloH[contDeH]=H;
        
   printf("\nla iteracion %d  para  %lu H es:  \n", contDeH,arregloH[contDeH] );

        contDeH++;
    
        /*BUscamos un H pequeño: la perturbación no debe ser 
        tan fuerte. Por tal motivo, se aplica una operación módulo.*/
        H%=256;
        
        /*Ahora, elegimos un valor para epsilon.*/
        epsilon = (H & 1)?1:-1;
        
        //printf("%d epsilon  %lu \n", epsilon, iteraciones-1); 
     
 

   } while (iteraciones < IT);

   
   /*Escribimos la informacion.*/
   fwrite(Xtotal,4,80000,archivobin); 
   fwrite(arregloH,4,20000,archivoH); 

   if(!fclose(archivobin)){
      printf( "\nArchivo binario de Mapas cerrado\n" );
   }
   else{
      printf( "\nError: Archivo binario de Mapas no cerrado \n" );
      return 1;
   }
   
   if(!fclose(archivoH)){
      printf( "\nArchivo binario DE H cerrado\n" );
   }
   else{
      printf( "\nError: Archivo binario DE H no cerrado \n" );
      return 1;
   }
   
return 0;
}
\end{verbatim}

\subsection{Código del criterio 2.}
\begin{verbatim}
#include <stdio.h>
#include <stdlib.h>
#include <math.h>


#define RENYI_MAP(var, parametro, j) ((var)*(parametro)+((var)>>(j)))
#define  MAX 4294967295 /*El valor de 2^32-1.*/
#define TAMANIOCICLO 4294967296 /*EL valor de 2^32.*/
#define noMapas 4 /*NUmero de mapas a utilizar.*/
#define ITtotales 80000 /*Iteraciones totales para NIST.*/
#define tamanioH 20000 /*Tamanio del arreglo de H.*/

/*
 File:   main.c
 Author: daniel
 El siguiente programa es un ejemplo de cuatro mapas caoticos RENYI acoplados.
 Se utiliza la suma para lograr el acoplado.
 Para declarar nuestras variables, utilizamos los siguientes componentes:
 -Un arreglo que guarda el valor de los parametros para cada mapa.
 -Un arreglo que guarda el valor de los valores calculados para cada mapa.
*/

int main(){
 
   /*Declaramos los arreglos que vamos a utilizar para guardar esto.*/
   unsigned long Xn[noMapas];
   unsigned long Xtotal[ITtotales]; 
   unsigned long parametros[noMapas];
   unsigned long arregloH[tamanioH];
   unsigned int i;
   int epsilon;
   FILE *  archivobin; 
   FILE *  archivoH;
  
   unsigned long H=0; 
   unsigned int j=9; /*No mas de 16.*/
   unsigned long iteraciones=0;
   unsigned long IT = 80000;

   /*Apertura del fichero de destino, para escritura en binario.*/
   archivobin = fopen ("binarioXORmod.dat", "wb");
   if (archivobin==NULL)
   {
   perror("No se puede abrir binarioXORmod.dat");
   return -1;
   }
   
    /* Apertura del archivo H para su escritura en binario.*/
   archivoH = fopen ("HdeXOR.dat", "wb");
   if (archivoH==NULL)
   {
   perror("No se puede abrir HdeXOR.dat");
   return -1;
   }
   
   /*Inicializamos nuestros parametros, en este punto se aplica el uso
   de una llave, en este ejemplo todavia no elegimos una. Tambien,
   los parametros son fijos en este ejemplo.*/
   parametros[0]=131071;
   Xn[0]=653;
   parametros[1]=104729;
   Xn[1]=769;
   parametros[2]=524287;
   Xn[2]=227;
   parametros[3]=65537;
   Xn[3]=823;
   int contDeH=0;
                   
   /*Primero, hacemos un ciclo inicial para calcular un nuevo valor para cada
   uno de los mapas y calcular, por primera vez, el resultado de la operacion
   XOR.*/
   for( i =0; i<noMapas; i++){
       Xn[i]= RENYI_MAP(Xn[i],parametros[i],j);
       H^=Xn[i];
   }
   
   arregloH[contDeH]=H; 
   contDeH++;
    
   /*Tambien al primer H se le aplica el mod.*/
   H%=256; 
 
   /*Elegimos un epsilon aleatorio entre 1 y 16 (para operaciones de 32 bits).
   En este caso, elegimos uno que este entre 0 y 15.*/
   epsilon = 1;   
   
   unsigned int k;
   unsigned long newH;

   do {
        
        newH = 0;
        for(k=0;k<noMapas; k++){            
            Xn[k]= RENYI_MAP(Xn[k],parametros[k],j) + epsilon*H;  
            Xtotal[iteraciones++] = Xn[k];
            newH^=Xn[k];      
        }
        H = newH;
        arregloH[contDeH]=H;
        printf("\nla iteracion %d  para  %lu H es:  \n", contDeH,arregloH[contDeH] );
        contDeH++;
        /*BUscamos un H pequeño: la perturbación no debe ser 
        tan fuerte. Por tal motivo, se aplica una operación módulo.*/
        H%=256; 
        
        /*Ahora, elegimos un valor para epsilon.*/
        epsilon = (H & 1)?1:-1;
    
   } while (iteraciones < IT);
   
   /*Escribimos la informacion.*/
   fwrite(Xtotal,4,80000,archivobin); 
   fwrite(arregloH,4,20000,archivoH); 
  
   if(!fclose(archivobin)){
      printf( "\nArchivo binario cerrado\n" );
   }
   else{
      printf( "\nError: Archivo binario no cerrado \n" );
      return 1;
   }
   
   if(!fclose(archivoH)){
     printf( "\nArchivo binario DE H cerrado\n" );
   }
   else{
      printf( "\nError: Archivo binario DE H no cerrado \n" );
      return 1;
   }
   
return 0;
}
\end{verbatim}

\subsection{Código del criterio 3.}
\begin{verbatim}
#include <stdio.h>
#include <stdlib.h>
#include <math.h>


#define RENYI_MAP(var, parametro, j) ((var)*(parametro)+((var)>>(j)))
#define  MAX 4294967295 /*El valor de 2^32-1.*/
#define TAMANIOCICLO 4294967296 /*EL valor de 2^32.*/
#define noMapas 4 /*NUmero de mapas a utilizar.*/
#define ITtotales 80000 /*Iteraciones totales para NIST.*/
#define tamanioH 20000 /*Tamanio del arreglo de H.*/

/*
 File:   main.c
 Author: daniel
 El siguiente programa es un ejemplo de cuatro mapas caoticos RENYI acoplados.
 Para declarar nuestras variables, utilizamos los siguientes componentes:
 -Un arreglo que guarda el valor de los parametros para cada mapa.
 -Un arreglo que guarda el valor de los valores calculados para cada mapa.
*/


int main(){
 
   
   /*Declaramos los arreglos que vamos a utilizar para guardar esto.*/
   unsigned long Xn[noMapas];
   unsigned long Xtotal[ITtotales]; 
   unsigned long parametros[noMapas];
   unsigned long arregloH[tamanioH];
   unsigned int i;
   unsigned int epsilon;
   unsigned long gamma;
   unsigned long gammaComp;
   FILE *  archivobin; 
   FILE *  archivoH;
   
   unsigned long H=0; 
   unsigned int j=9; /*No mas de 16.*/
   unsigned long iteraciones=0;
   unsigned long IT = 80000;

  /* Apertura del fichero de destino, para escritura en binario.*/
   archivobin = fopen ("binarioXORcomp.dat", "wb");
   if (archivobin==NULL)
   {
   perror("No se puede abrir binarioXORcomp.dat");
   return -1;
   }
   
   /* Apertura del archivo H para su escritura en binario.*/
   archivoH = fopen ("HdeXORcomp.dat", "wb");
   if (archivoH==NULL)
   {
   perror("No se puede abrir HdeXORcomp.dat");
   return -1;
   }
   
   /*Inicializamos nuestros parametros, en este punto se aplica el uso
    *de una llave, en este ejemplo todavia no elegimos una. Tambien,
    *los parametros son fijos en este ejemplo.*/
   parametros[0]=131071;
   Xn[0]=653;
   parametros[1]=104729;
   Xn[1]=769;
   parametros[2]=524287;
   Xn[2]=227;
   parametros[3]=65537;
   Xn[3]=823;
   int contDeH=0;
                   
   /*Primero, hacemos un ciclo inicial para calcular un nuevo valor para cada
   uno de los mapas y calcular, por primera vez, el resultado de la operacion
   XOR.*/
   for( i =0; i<noMapas; i++){
       Xn[i]= RENYI_MAP(Xn[i],parametros[i],j);
       H^=Xn[i];
   }
   
   arregloH[contDeH]=H; 
   contDeH++;
 
   /*Elegimos un epsilon aleatorio entre 1 y 16 (para operaciones de 32 bits).
   En este caso, elegimos uno que este entre 0 y 15.*/
   epsilon = 5;   
  
   gamma= pow(2,epsilon);
   gamma-=1;
   gammaComp= MAX-gamma;
    
   unsigned int k;
   unsigned long newH;

   do {
        
        newH = 0;
        for(k=0;k<noMapas; k++){            
            Xn[k]= gammaComp^RENYI_MAP(Xn[k],parametros[k],j)+ gamma^H;
            Xtotal[iteraciones++] = Xn[k];
            newH^=Xn[k];      
        }
        H = newH;
        arregloH[contDeH]=H;
        printf("\nla iteracion %d  para  %lu H es:  \n", contDeH,arregloH[contDeH] );
        contDeH++;
        /*Ahora, elegimos un valor para epsilon entre 1 y 8 bits.*/
        epsilon = (H % 8) + 1;  

   } while (iteraciones < IT);
 

   /*Escribimos la informacion.*/
   fwrite(Xtotal,4,80000,archivobin); 
   fwrite(arregloH,4,20000,archivoH); 

   if(!fclose(archivobin)){
      printf( "\nArchivo binario cerrado\n" );
   }
   else{
      printf( "\nError: Archivo binario no cerrado \n" );
      return 1;
   }  

   if(!fclose(archivoH)){
     printf( "\nArchivo binario DE H cerrado\n" );
   }
   else{
      printf( "\nError: Archivo binario DE H no cerrado \n" );
      return 1;
   }
   

return 0;
}
\end{verbatim}

\end{document}

