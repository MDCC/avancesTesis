%%%%%%%%%%%%%%%%%%%%%%%%%%%%%%%%%%%%%%%%%%%%%%%%%%%%%%%%%%%%%%%%%%
%%%%%%%% ICML 2013 EXAMPLE LATEX SUBMISSION FILE %%%%%%%%%%%%%%%%%
%%%%%%%%%%%%%%%%%%%%%%%%%%%%%%%%%%%%%%%%%%%%%%%%%%%%%%%%%%%%%%%%%%

% Use the following line _only_ if you're still using LaTeX 2.09.
%\documentstyle[icml2013,epsf,natbib]{article}
% If you rely on Latex2e packages, like most moden people use this:
\documentclass{article}

% For figures
\usepackage{graphicx} % more modern
%\usepackage{epsfig} % less modern
\usepackage{subfigure} 

% For citations
\usepackage{natbib}

% For algorithms
\usepackage{algorithm}
\usepackage{algorithmic}

% As of 2011, we use the hyperref package to produce hyperlinks in the
% resulting PDF.  If this breaks your system, please commend out the
% following usepackage line and replace \usepackage{icml2013} with
% \usepackage[nohyperref]{icml2013} above.
\usepackage{hyperref}

% Packages hyperref and algorithmic misbehave sometimes.  We can fix
% this with the following command.
\newcommand{\theHalgorithm}{\arabic{algorithm}}

% Employ the following version of the ``usepackage'' statement for
% submitting the draft version of the paper for review.  This will set
% the note in the first column to ``Under review.  Do not distribute.''
\usepackage{icml2013} 
% Employ this version of the ``usepackage'' statement after the paper has
% been accepted, when creating the final version.  This will set the
% note in the first column to ``Proceedings of the...''
% \usepackage[accepted]{icml2013}


% The \icmltitle you define below is probably too long as a header.
% Therefore, a short form for the running title is supplied here:
\icmltitlerunning{Submission and Formatting Instructions for ICML 2013}

\begin{document} 

\onecolumn
\icmltitle{Tabla de resultados de pruebas de aleatoriadad a la secuencia \\
           de datos ''fff'' de 2000,000 de bits. 
           }




% You may provide any keywords that you 
% find helpful for describing your paper; these are used to populate 
% the "keywords" metadata in the PDF but will not be shown in the document
\icmlkeywords{boring formatting information, machine learning, ICML}

\vskip 0.3in





 

\begin{table}[!h]
\caption{Resultados de las pruebas de aleatoriedad NIST a los datos ''fff''.}
\label{sample-table}
\vskip 0.15in
\begin{center}
\begin{small}
\begin{sc}
\begin{tabular}{lccr}
\hline
\abovespace\belowspace
Prueba Aplicada &  P-Valor & Exito? \\
\hline
\abovespace
Aproximate Entropy    &  0.303647   & $\surd$ \\
\abovespace
Block Frecuency  &  0.450408  &  $\surd$  \\
\abovespace
Cumulative Sums    &   Forward test: 0.777054, Reverse test: 0.671857   & $\surd$ \\
\abovespace
FFT    &   0.617327 &   $\surd$      \\
\abovespace
Frecuency     &  0.820988  &  $\surd$   \\
\abovespace
Linear Complexity      & 0.784292 & $\surd$ \\
\abovespace
Longest Run      &  0.795471 &    $\surd$      \\
\abovespace
Non Overlapping Template      & P-valores aceptados: 148 de 148    &     $\surd$          \\
\abovespace
Overlapping Template      & 0.642080 &        $\surd$       \\
\abovespace
Random Excursions      & P-valores aceptados: 8 de 8  &     $\surd$          \\
\abovespace
Random Excursions Variant & P-valores aceptados: 18 de 18  &    $\surd$        \\
\abovespace
Rank &    0.707380      &       $\surd$      \\
\abovespace
Runs &        0.265730    &     $\surd$        \\
\abovespace
Serial &     P-valores aceptados: 2 de 2     &     $\surd$        \\
\abovespace
Universal &     0.696176  &   $\surd$            \\

\hline
\abovespace


\end{tabular}
\end{sc}
\end{small}
\end{center}
\vskip -0.1in
\end{table}




\end{document} 


% This document was modified from the file originally made available by
% Pat Langley and Andrea Danyluk for ICML-2K. This version was
% created by Lise Getoor and Tobias Scheffer, it was slightly modified  
% from the 2010 version by Thorsten Joachims & Johannes Fuernkranz, 
% slightly modified from the 2009 version by Kiri Wagstaff and 
% Sam Roweis's 2008 version, which is slightly modified from 
% Prasad Tadepalli's 2007 version which is a lightly 
% changed version of the previous year's version by Andrew Moore, 
% which was in turn edited from those of Kristian Kersting and 
% Codrina Lauth. Alex Smola contributed to the algorithmic style files.  
