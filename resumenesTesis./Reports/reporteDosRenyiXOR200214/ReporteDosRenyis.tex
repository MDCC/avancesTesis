
%%\documentclass[preprint,12pt,3p]{elsarticle}
\documentclass[12pt,3p]{elsarticle}

\usepackage{amssymb}

\usepackage[utf8]{inputenc}
\usepackage[spanish]{babel}
\usepackage{graphicx}
\usepackage{amsmath}
\usepackage{subfigure} 
\usepackage{float}
% For algorithms
\usepackage{algorithm}
\usepackage{algorithmic}





\journal{--}

\begin{document}

\begin{frontmatter}

\title{Operación XOR en dos Renyis Acoplados Y aplicación de pruebas NIST.}



\author{Marcos Daniel Calderón Calderón}



\ead{marcos.calderon@cimat.mx}




\begin{abstract}
En este reporte se presentan a pruebas NIST de dos Mapas Renyi cuyos resultados son combinados por medio de una operación XOR.
\end{abstract}



\end{frontmatter}

\section{Ejemplo 1.}

\subsection{Procedimiento.}
EL mapa caótico Renyi tiene la siguiente forma:

\begin{equation}
f(k)=  \left(  q2^{n-i}k +  \lfloor \frac{k}{2^{j}} \rfloor   \right) \mod{ 2^{n}}
\end{equation}

para que el mapa sea invertible, es necesario que se cumpla la siguiente condición:

\begin{equation}
0 < i  = j \leq n
\end{equation}

por lo tanto, hemos elegido los siguientes parámetros:



\begin{itemize}
\item Mapa 1: $i = j = 9$.


\item Mapa 2: $i = j = 13$.
\end{itemize}

Ahora, es necesario elegir un valor de $q$ adecuado, en este ejemplo, utilizamos los siguientes:

\begin{itemize}
\item Mapa 1: $q = 13.$
\item Mapa 2: $q = 19.$
\end{itemize}


Ahora, es necesario calcular el parámetro que se forma de la expresión:  
\begin{equation}
\beta = q 2^{n-i}
\end{equation}
con base en lo anterior, se encontraron los siguientes parámetros para el mapa 1 y el mapa 2 respectivamente:

\begin{itemize}
\item $\beta_{1}=109051904$.
\item $\beta_{2}=9961472$.
\end{itemize}



 
También es importante elegir el tipo de dato que se va a utilizar para almacenar los valores generados por los mapas, en este caso se eligió el tipo de dato \textbf{unsigned long} (se utilizó un equipo con arquitectura de 32 bits, donde este tipo de dato tiene un tamaño de 32 bits).

Se generaron 80,000 valores  a la hora de relacionar los valores de los dos mapas con la operación XOR. Como cada valor está formado de 32 bits, se  obtiene un total de $2,560,000$ bits para la aplicación de pruebas NIST.



\subsection{Resultados.}

Se utilizó el siguiente código para la ejecución de las pruebas NIST:

Se ejecutó el siguiente código:

\begin{itemize}
\item \textbf{./assess 2560000}
\item User Prescribed Input File: \textbf{dosRenyis.dat}
\item    Enter 0 if you DO NOT want to apply all of the
         statistical tests to each sequence and 1 if you DO. Enter chice: \textbf{1}
                  
\item  How many bitstreams? \textbf{1}

\item Input File Format:
    [0] ASCII - A sequence of ASCII 0's and 1's
    [1] Binary - Each byte in data file contains 8 bits of data

   Select input mode:  \textbf{1}
\end{itemize}


LOs resultados obtenidos son los siguientes:

 

\begin{table}[H]
\caption{Resultados de las pruebas de aleatoriedad NIST a los datos dosRenyis.dat .}
\label{sample-table}
\vskip 0.15in
\begin{center}
\begin{small}
\begin{sc}
\begin{tabular}{lccr}
\hline

Prueba Aplicada &  P-Valor & Exito? \\
\hline

Aproximate Entropy    &   0.751203  & $\surd$ \\

Block Frecuency  &   0.818897  &  $\surd$  \\

Cumulative Sums    &   Forward test:0.557353, Reverse test: 0.508345  & $\surd$ \\

FFT    &   0.535638 &   $\surd$      \\

Frecuency     &  0.412929  &  $\surd$   \\

Linear Complexity      &  0.893675  & $\surd$ \\

Longest Run      &  0.864540 &    $\surd$      \\

Non Overlapping Template      & P-valores aceptados: 148 de 148    &     $\surd$          \\

Overlapping Template      &  0.583516  &        $\surd$       \\

Random Excursions      & P-valores aceptados:  8 de 8  &    $\surd$      \\

Random Excursions Variant & P-valores aceptados:  18 de 18 &     $\surd$    \\

Rank &    0.416165  &      $\surd$      \\

Runs &    0.052786 &     $\surd$        \\

Serial &     P-valores aceptados: 2 de 2    &     $\surd$        \\

Universal &      0.509933 &   $\surd$            \\

\hline



\end{tabular}
\end{sc}
\end{small}
\end{center}
\vskip -0.1in
\end{table}




\section{Ejemplo 2.}

\subsection{Procedimiento.}
Para el ejemplo 2, se eligieron los siguientes parámetros:

\begin{itemize}
\item Mapa 1: $i = j = 5$.


\item Mapa 2: $i = j = 14$.
\end{itemize}

Ahora, es necesario elegir un valor de $q$ adecuado, en este ejemplo, utilizamos los siguientes:

\begin{itemize}
\item Mapa 1: $q = 29.$
\item Mapa 2: $q = 31.$
\end{itemize}


Ahora, es necesario calcular el parámetro que se forma de la expresión:  
\begin{equation}
\beta = q 2^{n-i}
\end{equation}
con base en lo anterior, se encontraron los siguientes parámetros para el mapa 1 y el mapa 2 respectivamente:

\begin{itemize}
\item $\beta_{1}=3892314112$.
\item $\beta_{2}=8126464$.
\end{itemize}


\subsection{Resultados.}

Se utilizó el siguiente código para la ejecución de las pruebas NIST:

Se ejecutó el siguiente código:

\begin{itemize}
\item \textbf{./assess 2560000}
\item User Prescribed Input File: \textbf{dosRenyis2.dat}
\item    Enter 0 if you DO NOT want to apply all of the
         statistical tests to each sequence and 1 if you DO. Enter chice: \textbf{1}
                  
\item  How many bitstreams? \textbf{1}

\item Input File Format:
    [0] ASCII - A sequence of ASCII 0's and 1's
    [1] Binary - Each byte in data file contains 8 bits of data

   Select input mode:  \textbf{1}
\end{itemize}


LOs resultados obtenidos son los siguientes:

 

\begin{table}[H]
\caption{Resultados de las pruebas de aleatoriedad NIST a los datos dosRenyis2.dat .}
\label{sample-table}
\vskip 0.15in
\begin{center}
\begin{small}
\begin{sc}
\begin{tabular}{lccr}
\hline

Prueba Aplicada &  P-Valor & Exito? \\
\hline

Aproximate Entropy    &   0.427962 & $\surd$ \\

Block Frecuency  &   0.766879  &  $\surd$  \\

Cumulative Sums    &   Forward test:0.368550, Reverse test: 0.541795  & $\surd$ \\

FFT    &   0.613759 &   $\surd$      \\

Frecuency     &  0.298921 &  $\surd$   \\

Linear Complexity      &  0.609575 & $\surd$ \\

Longest Run      &  0.080531 &    $\surd$      \\

Non Overlapping Template      & P-valores aceptados: 148 de 148    &     $\surd$          \\

Overlapping Template      &  0.992376  &        $\surd$       \\

Random Excursions      & P-valores aceptados:  8 de 8  &    $\surd$      \\

Random Excursions Variant & P-valores aceptados:  18 de 18 &     $\surd$    \\

Rank &    0.827165  &      $\surd$      \\

Runs &    0.260303 &     $\surd$        \\

Serial &     P-valores aceptados: 2 de 2    &     $\surd$        \\

Universal &      0.652881 &   $\surd$            \\

\hline



\end{tabular}
\end{sc}
\end{small}
\end{center}
\vskip -0.1in
\end{table}




\section{Conclusiones.}
Si en los mapas acoplados $j_{1}= j_{2}$, puede ocurrir que no se aprueben las pruebas NIST, pero si ocurre que  $j_{1} \neq j_{2}$, los p-valores son muy buenos y las pruebas NIST son exitosas. Se obtienen buenos valores para las pruebas NIST si se eligen de manera decuada los parámetros.
\section{Anexos.}

\subsection{Código del ejemplo 1.}
\begin{verbatim}
#include <stdio.h>
#include <stdlib.h>
#include <math.h>


#define RENYI_MAP(var, parametro, j) ((var)*(parametro)+((var)>>(j)))
#define ITtotales 80000 /*Iteraciones totales para NIST.*/


/*
 File:   main.c
 Author: daniel
En este programa, se ejecutan de manera simultánea dos mapas de tipo Renyi,
 estos no
son acoplados, las salidas de estos mapas se utilizan para una 
operacion XOR.
*/



unsigned long calcular_parametro(unsigned long q, unsigned int n, unsigned int i){    
    unsigned long potencia =1;
    unsigned int j;
    /*Calculo de la potencia (calculamos 2(a la)( n-i) ): */
    for(j =0; j<(n-i); j++){
        potencia*=2;
    }

    /*Ahora, lo que hacemos es la operacion q*2(a la )(n-i): */
    unsigned long parametro = q*(potencia);
    printf("  EL parametro es: %lu \n \n \n",parametro);
    return parametro;
}

int main(){
 
   
   /*Declaramos los arreglos que vamos a utilizar para guardar esto.
    EL tipo de dato será
   unsigned long, cuyo tamaño es de 32 bits
    (arquitectura de 32 bits).*/
    
   unsigned long Xtotal[ITtotales]; 
   FILE *  archivobin; 
   unsigned long Xn1 = 7;
   unsigned long Xn2 = 9;
   unsigned int n = 32;

   /*Valores de i, j y q para el mapa 1.*/
   unsigned int i1=9; 
   unsigned int j1=9; 
   unsigned long q1=13;
   /*Valores de i, j y q para el mapa 2.*/
   unsigned int i2=13; 
   unsigned int j2=13; 
   unsigned long q2=19;
 
   /*Calculo de los parametros para cada mapa: 1 y 2 respectivamente.*/
   unsigned long param1 =calcular_parametro(q1, n, i1);
   unsigned long param2 =calcular_parametro(q2, n, i2);
 
   unsigned int iteraciones=0;
   unsigned int IT = 80000;

  /* Apertura del fichero de destino, para escritura en binario.*/
   archivobin = fopen ("dosRenyis.dat", "wb");
   if (archivobin==NULL)
   {
   perror("No se puede abrir dosRenyis.dat");
   return -1;
   }
   
   printf("\n\n\n Operacion XOR en dos Renyis \n EL 
   tamanio de unsigned long en maquina de 32 bits es:  %d", sizeof(long));



   while (iteraciones < IT) {

            Xn1= RENYI_MAP(Xn1,param1,j1);
            Xn2= RENYI_MAP(Xn2,param2,j2);
            Xtotal[iteraciones++] = Xn1^Xn2;
   } 
 

   /*Escribimos la informacion.*/
   fwrite(Xtotal,4,80000,archivobin); 

   if(!fclose(archivobin)){
      printf( "\nArchivo binario cerrado\n" );
   }
   else{
      printf( "\nError: Archivo binario no cerrado \n" );
      return 1;
   }  

  

return 0;
}
\end{verbatim}

\subsection{Código del criterio 2.}
\begin{verbatim}
#include <stdio.h>
#include <stdlib.h>
#include <math.h>


#define RENYI_MAP(var, parametro, j) ((var)*(parametro)+((var)>>(j)))
#define ITtotales 80000 /*Iteraciones totales para NIST.*/


/*
 File:   main.c
 Author: daniel
En este programa, se ejecutan de manera simultánea dos mapas de tipo Renyi, estos no
son acoplados, las salidas de estos mapas se utilizan para una operacion XOR.
*/



unsigned long calcular_parametro(unsigned long q, unsigned int n, unsigned int i){    
    unsigned long potencia =1;
    unsigned int j;
    /*Calculo de la potencia (calculamos 2(a la)( n-i) ): */
    for(j =0; j<(n-i); j++){
        potencia*=2;
    }

    /*Ahora, lo que hacemos es la operacion q*2(a la )(n-i): */
    unsigned long parametro = q*(potencia);
    printf("  EL parametro es: %lu \n \n \n",parametro);
    return parametro;
}

int main(){
 
   
   /*Declaramos los arreglos que vamos a utilizar para guardar esto
   . EL tipo de dato será
   unsigned long, cuyo tamaño es de 32 bits (arquitectura de 32 bits).*/
   unsigned long Xtotal[ITtotales]; 
   FILE *  archivobin; 
   unsigned long Xn1 = 13;
   unsigned long Xn2 = 19;
   unsigned int n = 32;

   /*Valores de i, j y q para el mapa 1.*/
   unsigned int i1=5; 
   unsigned int j1=5; 
   unsigned long q1=29;
   /*Valores de i, j y q para el mapa 2.*/
   unsigned int i2=14; 
   unsigned int j2=14; 
   unsigned long q2=31;
 
   /*Calculo de los parametros para cada mapa: 1 y 2 respectivamente.*/
   unsigned long param1 =calcular_parametro(q1, n, i1);
   unsigned long param2 =calcular_parametro(q2, n, i2);
 
   unsigned int iteraciones=0;
   unsigned int IT = 80000;

  /* Apertura del fichero de destino, para escritura en binario.*/
   archivobin = fopen ("dosRenyis2.dat", "wb");
   if (archivobin==NULL)
   {
   perror("No se puede abrir dosRenyis2.dat");
   return -1;
   }
   
   printf("\n\n\n Operacion XOR en dos Renyis \n EL tamanio de 
   unsigned long en maquina de 32 bits es:  %d", sizeof(long));



   while (iteraciones < IT) {

            Xn1= RENYI_MAP(Xn1,param1,j1);
            Xn2= RENYI_MAP(Xn2,param2,j2);
            Xtotal[iteraciones++] = Xn1^Xn2;
 
   } 
 

   /*Escribimos la informacion.*/
   fwrite(Xtotal,4,80000,archivobin); 

   if(!fclose(archivobin)){
      printf( "\nArchivo binario cerrado\n" );
   }
   else{
      printf( "\nError: Archivo binario no cerrado \n" );
      return 1;
   }  

  

return 0;
}
\end{verbatim}



\end{document}

