
%%\documentclass[preprint,12pt,3p]{elsarticle}
\documentclass[12pt,3p]{elsarticle}

\usepackage{amssymb}

\usepackage[utf8]{inputenc}
\usepackage[spanish]{babel}
\usepackage{graphicx}
\usepackage{amsmath}
\usepackage{subfigure} 
\usepackage{float}
% For algorithms
\usepackage{algorithm}
\usepackage{algorithmic}





\journal{--}

\begin{document}

\begin{frontmatter}

\title{Implementación de Renyi Maps acoplados y aplicación de pruebas NIST.}



\author{Marcos Daniel Calderón Calderón}



\ead{marcos.calderon@cimat.mx}




\begin{abstract}
En este reporte, se explica a detalle la implementación de varios mapas caóticos acoplados. En este caso, se utilizará el mapa Renyi. 
\end{abstract}



\end{frontmatter}

\section{Introducción: Sistemas acoplados.}

Un sistema acoplado puede estar representado por medio de ecuaciones diferenciales o discretas. En este tipo de sistemas, las razones de cambio dependen de más variables. Un ejemplo sencillo puede ser el siguiente sistema de ecuaciones diferenciales:

\begin{equation}
\frac{dC}{dt}= \alpha C - \beta CZ \quad
\frac{dZ}{dt}= - \gamma Z + \delta CZ
\end{equation}

Un ejemplo de sistema discreto acoplado es el siguiente:

\begin{equation}
X_{i,j}= (1 -\epsilon )f(X_{i,j-1})+ \epsilon H(X_{i,j-1},...,X_{N,j-1})
\end{equation}

donde:
\begin{equation}
H(X_{i,j-1},...,X_{N,j-1}) = \sum_{i=1}^{N}w_{i}X_{i,j-1}.
\end{equation}

El indice $i$ representa cada una de los $N$ mapas que se van a acoplar, el índice $j$ representa a   las evaluaciones anteriores de cada uno de los mapas.

\section{Criterios utilizados para el acoplamiento de mapas.}
Se utilizaron diversas maneras para el acoplamiento de los mapas; a continuación, explicamos cada una de ellas.

\subsection{Criterio 1.}

El primer criterio utilizado es el siguiente:

\begin{equation}
X_{i,j}= f(X_{i,j-1})+ \epsilon  H(X_{i,j-1},...,X_{N,j-1})
\end{equation}

donde:
\begin{equation}
H(X_{i,j-1},...,X_{N,j-1}) = \sum_{i=1}^{N}(X_{i,j-1}\mod 256).
\end{equation}
además, $\epsilon$ es un valor aleatorio del conjunto: ${\{-1, 0, 1 \}}$




\subsection{Criterio 2.}

El segundo criterio utilizado es el siguiente:

\begin{equation}
X_{i,j}= f(X_{i,j-1})+ \epsilon  H(X_{i,j-1},...,X_{N,j-1})
\end{equation}

donde:
\begin{equation}
H(X_{i,j-1},...,X_{N,j-1}) = \bigoplus _{i=1}^{N}(X_{i,j-1}).
\end{equation}
donde $\bigoplus$ representa la operación $XOR$, además $\epsilon$ es un valor aleatorio del conjunto: ${\{-1, 0, 1 \}}$



\subsection{Criterio 3.}

El tercer criterio utilizado es el siguiente:

\begin{equation}
X_{i,j}= \hat{\gamma} \bigoplus f(X_{i,j-1})+ \gamma \bigoplus H(X_{i,j-1},...,X_{N,j-1})
\end{equation}

donde:
\begin{equation}
H(X_{i,j-1},...,X_{N,j-1}) = \bigoplus _{i=1}^{N}(X_{i,j-1}).
\end{equation}
donde $\bigoplus$ representa la operación $XOR$, además para el cálculo del valor de $\gamma$, necesitamos un $\epsilon$ que será un valor aleatorio donde $1 \geq \epsilon \geq 32$, el rango es establecido de acuerdo al tipo de dato utilizado cuando se traduce el sistema a un lenguaje de programación elegido. Ahora, podemos calcular $\gamma$ de la siguiente manera:

\begin{equation}
\gamma = 2^{\epsilon}-1 \quad  
\end{equation}

\begin{equation}
\hat{\gamma} = M - \gamma
\end{equation}

donde $M = 2^{32}-1$.


\section{Detalles de implementación.}
A continuación, se especifican algunos detalles importantes a tomar en cuenta a la hora de la implementación de los mapas acoplados:

\begin{itemize}

\item El lenguaje de programación utilizado será \textbf{C}.
\item Se trabajará con el tipo de dato \textbf{unsigned long}; por lo tanto, se dispone de 32 bits (4 bytes) para la representación de los datos. Por lo tanto, tenemos un rango de $(0 - 4,294,967,295=2^{32} - 1) $.

\item Se buscaron 80000 valores en la ejecución del programa; además, cada valor está compuesto de 32 bits cuando se guarda la información en un archivo binario. Por las características mencionadas anteriormente, se pueden leer \textbf{2,560,000 bits} para la aplicación de las pruebas NIST.

\item Para equipos de cómputo de 32 bits, se utilizó el tipo de dato \textbf{unsigned long}, que está conformado de 4 bytes. Si el equipo de cómputo utilizado es de 64 bits, se recomienta utilizar el tipo de dato \textbf{unsigned int}, todavía no hay un estándar definido para el tamaño de los tipos de datos en los equipos de 64 bits.
\end{itemize}





\section{Resultados.}

\subsection{Aplicación de las pruebas NIST a los mapas acoplados del Criterio 1.}
Recordemos que el criterio 1 era el siguiente:

\begin{equation}
X_{i,j}= f(X_{i,j-1})+ \epsilon  H(X_{i,j-1},...,X_{N,j-1})
\end{equation}

donde:
\begin{equation}
H(X_{i,j-1},...,X_{N,j-1}) = \sum_{i=1}^{N}(X_{i,j-1}\mod 256).
\end{equation}
además, $\epsilon$ es un valor aleatorio del conjunto: ${\{-1, 0, 1 \}}$



Se ejecutó el siguiente código:

\begin{itemize}
\item \textbf{./assess 2560000}
\item User Prescribed Input File: \textbf{binarioSUMA.dat}
\item    Enter 0 if you DO NOT want to apply all of the
         statistical tests to each sequence and 1 if you DO. Enter chice: \textbf{1}
                  
\item  How many bitstreams? \textbf{1}

\item Input File Format:
    [0] ASCII - A sequence of ASCII 0's and 1's
    [1] Binary - Each byte in data file contains 8 bits of data

   Select input mode:  \textbf{1}
\end{itemize}


Se obtuvieron los siguientes resultados:

\begin{table}[!h]
\caption{Resultados de las pruebas de aleatoriedad NIST a los datos binarioSUMA.dat .}
\label{sample-table}
\vskip 0.15in
\begin{center}
\begin{small}
\begin{sc}
\begin{tabular}{lccr}
\hline

Prueba Aplicada &  P-Valor & Exito? \\
\hline

Aproximate Entropy    &   0.241029  & $\surd$ \\

Block Frecuency  &  0.464753 &  $\surd$  \\

Cumulative Sums    &   Forward test:  0.628651, Reverse test: 0.848525  & $\surd$ \\

FFT    &   0.149985 &   $\surd$      \\

Frecuency     &  0.625019 &  $\surd$   \\

Linear Complexity      &  0.595287  & $\surd$ \\

Longest Run      &   0.511949  &    $\surd$      \\

Non Overlapping Template      & P-valores aceptados: 145 de 148    &     $\surd$          \\

Overlapping Template      &  0.037821  &        $\surd$       \\

Random Excursions      & P-valores aceptados: 8 de 8  &     $\surd$          \\

Random Excursions Variant & P-valores aceptados: 18 de 18  &    $\surd$        \\

Rank &    0.111388    &       $\surd$      \\

Runs &        0.957015  &     $\surd$        \\

Serial &     P-valores aceptados: 2 de 2    &     $\surd$        \\

Universal &     0.136601 &   $\surd$            \\

\hline



\end{tabular}
\end{sc}
\end{small}
\end{center}
\vskip -0.1in
\end{table}


\subsection{Aplicación de las pruebas NIST a los mapas acoplados del Criterio 2.}

El criterio 2 era el siguiente:
\begin{equation}
X_{i,j}= f(X_{i,j-1})+ \epsilon  H(X_{i,j-1},...,X_{N,j-1})
\end{equation}

donde:
\begin{equation}
H(X_{i,j-1},...,X_{N,j-1}) = \bigoplus _{i=1}^{N}(X_{i,j-1}).
\end{equation}
donde $\bigoplus$ representa la operación $XOR$, además $\epsilon$ es un valor aleatorio del conjunto: ${\{-1, 0, 1 \}}$


Para el criterio 2, se ejecutó el siguiente código:

\begin{itemize}
\item \textbf{./assess 2560000}
\item User Prescribed Input File: \textbf{binarioXOR.dat}
\item    Enter 0 if you DO NOT want to apply all of the
         statistical tests to each sequence and 1 if you DO. Enter chice: \textbf{1}
                  
\item  How many bitstreams? \textbf{1}

\item Input File Format:
    [0] ASCII - A sequence of ASCII 0's and 1's
    [1] Binary - Each byte in data file contains 8 bits of data

   Select input mode:  \textbf{1}
\end{itemize}


Se obtuvieron los siguientes resultados:

\begin{table}[!h]
\caption{Resultados de las pruebas de aleatoriedad NIST a los datos binarioXOR.dat .}
\label{sample-table}
\vskip 0.15in
\begin{center}
\begin{small}
\begin{sc}
\begin{tabular}{lccr}
\hline

Prueba Aplicada &  P-Valor & Exito? \\
\hline

Aproximate Entropy    &   0.858228  & $\surd$ \\

Block Frecuency  & 0.118270  &  $\surd$  \\

Cumulative Sums    &   Forward test: 0.021856, Reverse test: 0.035757  & $\surd$ \\

FFT    &   0.787496 &   $\surd$      \\

Frecuency     &  0.022608 &  $\surd$   \\

Linear Complexity      &  0.791809  & $\surd$ \\

Longest Run      &   0.378018  &    $\surd$      \\

Non Overlapping Template      & P-valores aceptados: 144 de 148    &     $\surd$          \\

Overlapping Template      &  0.457920 &        $\surd$       \\

Random Excursions      &  NOT APPLICABLE. &             \\

Random Excursions Variant & NOT APPLICABLE. &          \\

Rank &    0.278871   &       $\surd$      \\

Runs &    0.019569  &     $\surd$        \\

Serial &     P-valores aceptados: 2 de 2    &     $\surd$        \\

Universal &      0.249147 &   $\surd$            \\

\hline



\end{tabular}
\end{sc}
\end{small}
\end{center}
\vskip -0.1in
\end{table}


\subsection{Aplicación de las pruebas NIST a los mapas acoplados del Criterio 3.}

EL criterio 3 era el siguiente:
\begin{equation}
X_{i,j}= \hat{\gamma} \bigoplus f(X_{i,j-1})+ \gamma \bigoplus H(X_{i,j-1},...,X_{N,j-1})
\end{equation}

donde:
\begin{equation}
H(X_{i,j-1},...,X_{N,j-1}) = \bigoplus _{i=1}^{N}(X_{i,j-1}).
\end{equation}



Para el criterio 3, se ejecutó el siguiente código:

\begin{itemize}
\item \textbf{./assess 2560000}
\item User Prescribed Input File: \textbf{binarioXORcomp.dat}
\item    Enter 0 if you DO NOT want to apply all of the
         statistical tests to each sequence and 1 if you DO. Enter chice: \textbf{1}
                  
\item  How many bitstreams? \textbf{1}

\item Input File Format:
    [0] ASCII - A sequence of ASCII 0's and 1's
    [1] Binary - Each byte in data file contains 8 bits of data

   Select input mode:  \textbf{1}
\end{itemize}


Se obtuvieron los siguientes resultados:

\begin{table}[!h]
\caption{Resultados de las pruebas de aleatoriedad NIST a los datos binarioXORcomp.dat .}
\label{sample-table}
\vskip 0.15in
\begin{center}
\begin{small}
\begin{sc}
\begin{tabular}{lccr}
\hline

Prueba Aplicada &  P-Valor & Exito? \\
\hline

Aproximate Entropy    &    0.979174  & $\surd$ \\

Block Frecuency  &  0.283892 &  $\surd$  \\

Cumulative Sums    &   Forward test:  0.320736, Reverse test: 0.476676  & $\surd$ \\

FFT    &   0.990848 &   $\surd$      \\

Frecuency     &  0.286876 &  $\surd$   \\

Linear Complexity      &  0.142002 & $\surd$ \\

Longest Run      &    0.106050 &    $\surd$      \\

Non Overlapping Template      & P-valores aceptados: 145 de 148    &     $\surd$          \\

Overlapping Template      &  0.879647  &        $\surd$       \\

Random Excursions      & P-valores aceptados: 8 de 8  &     $\surd$          \\

Random Excursions Variant & P-valores aceptados: 18 de 18  &    $\surd$        \\

Rank &    0.398102    &       $\surd$      \\

Runs &        0.964539  &     $\surd$        \\

Serial &     P-valores aceptados: 2 de 2    &     $\surd$        \\

Universal &     0.603939  &   $\surd$            \\

\hline



\end{tabular}
\end{sc}
\end{small}
\end{center}
\vskip -0.1in
\end{table}


\section{Conclusiones.}
En general, los criterios 1 y 3 arrojan mejores resultados que el criterio 2. Si se tuviera que elegir entre el criterio 1 y el 3. Se puede concluir que el criterio 3 obtuvo un mejor desempeño de acuerdo a los p-valores obtenidos.

\section{Anexos.}
Se anexan los códigos obtenidos para cada cirterio utilizado.
\subsection{Criterio 1.}
\begin{verbatim}
#include <stdio.h>
#include <stdlib.h>
#include <math.h>


#define RENYI_MAP(var, parametro, j) ((var)*(parametro)+((var)>>(j)))

#define  MAX 4294967295 /*El valor de 2^32-1.*/
#define TAMANIOCICLO 4294967296 /*EL valor de 2^32.*/
#define noMapas 4 /*NUmero de mapas a utilizar.*/
#define ITtotales 80000 /*Iteraciones totales para NIST.*/



/*
 File:   main.c
 Author: daniel
 El siguiente programa es un ejemplo de cuatro mapas caoticos RENYI acoplados.
 Se utiliza la suma para lograr el acoplado.
 Para declarar nuestras variables, utilizamos los siguientes componentes:
 -Un arreglo que guarda el valor de los parametros para cada mapa.
 -Un arreglo que guarda el valor de los valores calculados para cada mapa.
*/


int main(){
 
   
   /*Declaramos los arreglos que vamos a utilizar para guardar esto.*/
   unsigned long Xn[noMapas];
   unsigned long Xtotal[ITtotales]; 
   unsigned long parametros[noMapas];
   unsigned int i;
   int epsilon;
   FILE *  archivobin; 
   

   unsigned long H=0; 
   unsigned int j=9; /*No mas de 16.*/
   unsigned long iteraciones=0;
   unsigned long IT = 80000;


  /* Apertura del fichero de destino, para escritura en binario.*/
   archivobin = fopen ("binarioSUMA.dat", "wb");
   if (archivobin==NULL)
   {
   perror("No se puede abrir binarioSUMA.dat");
   return -1;
   }
   


   /*Inicializamos nuestros parametros, en este punto se aplica el uso
   de una llave, en este ejemplo todavia no elegimos una. Tambi��n,
   los parametros son fijos en este ejemplo.*/
   parametros[0]=131071;
   Xn[0]=653;
   parametros[1]=104729;
   Xn[1]=769;
   parametros[2]=524287;
   Xn[2]=227;
   parametros[3]=65537;
   Xn[3]=823;
                   
   /*Primero, hacemos un ciclo inicial para calcular un nuevo valor para cada
   uno de los mapas y calcular, por primera vez, el resultado de la operacion
   XOR.*/
   for( i =0; i<noMapas; i++){
       Xn[i]= RENYI_MAP(Xn[i],parametros[i],j);
       H+=Xn[i]%256;
   }
   
 
   /*Elegimos un epsilon aleatorio entre 1 y 16 (para operaciones de 32 bits).
   En este caso, elegimos uno que este entre 0 y 15.*/
   epsilon = 1;   
  
      
   unsigned int k;
   unsigned long newH;

   do {
        
        newH = 0;
        for(k=0;k<noMapas; k++){            
            Xn[k]= RENYI_MAP(Xn[k],parametros[k],j) + epsilon*H;
            Xtotal[iteraciones++] = Xn[k];
             
            newH+=(Xn[k]%256);      
        }
        H = newH;
        /*Ahora, elegimos un valor para epsilon entre 1 y 8 bits.*/
        epsilon = (H % 3) - 1;
     
 

   } while (iteraciones < IT);

   /*Escribimos la informacion.*/
   fwrite(Xtotal,4,80000,archivobin); 

   if(!fclose(archivobin)){
      printf( "\nArchivo binario cerrado\n" );
   }
   else{
      printf( "\nError: Archivo binario no cerrado \n" );
      return 1;
   }
   
   

return 0;
}
\end{verbatim}

\subsection{Criterio 2.}
\begin{verbatim}
#include <stdio.h>
#include <stdlib.h>
#include <math.h>


#define RENYI_MAP(var, parametro, j) ((var)*(parametro)+((var)>>(j)))

#define  MAX 4294967295 /*El valor de 2^32-1.*/
#define TAMANIOCICLO 4294967296 /*EL valor de 2^32.*/
#define noMapas 4 /*NUmero de mapas a utilizar.*/
#define ITtotales 80000 /*Iteraciones totales para NIST.*/



/*
 File:   main.c
 Author: daniel
 El siguiente programa es un ejemplo de cuatro mapas caoticos RENYI acoplados.
 Se utiliza la suma para lograr el acoplado.
 Para declarar nuestras variables, utilizamos los siguientes componentes:
 -Un arreglo que guarda el valor de los parametros para cada mapa.
 -Un arreglo que guarda el valor de los valores calculados para cada mapa.
*/


int main(){

  
 
   /*Declaramos los arreglos que vamos a utilizar para guardar esto.*/
   unsigned long Xn[noMapas];
   unsigned long Xtotal[ITtotales]; 
   unsigned long parametros[noMapas];
   unsigned int i;
   int epsilon;
   FILE *  archivobin; 
  

   unsigned long H=0; 
   unsigned int j=9; /*No mas de 16.*/
   unsigned long iteraciones=0;
   unsigned long IT = 80000;

   /*Apertura del fichero de destino, para escritura en binario.*/
   archivobin = fopen ("binarioXOR.dat", "wb");
   if (archivobin==NULL)
   {
   perror("No se puede abrir binarioXOR.dat");
   return -1;
   }
   

   /*Inicializamos nuestros parametros, en este punto se aplica el uso
   de una llave, en este ejemplo todavia no elegimos una. Tambi��n,
   los parametros son fijos en este ejemplo.*/
   parametros[0]=131071;
   Xn[0]=653;
   parametros[1]=104729;
   Xn[1]=769;
   parametros[2]=524287;
   Xn[2]=227;
   parametros[3]=65537;
   Xn[3]=823;
                   
   /*Primero, hacemos un ciclo inicial para calcular un nuevo valor para cada
   uno de los mapas y calcular, por primera vez, el resultado de la operacion
   XOR.*/
   for( i =0; i<noMapas; i++){
       Xn[i]= RENYI_MAP(Xn[i],parametros[i],j);
       H^=Xn[i];
   }
   
 
   /*Elegimos un epsilon aleatorio entre 1 y 16 (para operaciones de 32 bits).
   En este caso, elegimos uno que este entre 0 y 15.*/
   epsilon = 1;   
  
      
   unsigned int k;
   unsigned long newH;

   do {
        
        newH = 0;
        for(k=0;k<noMapas; k++){            
            Xn[k]= RENYI_MAP(Xn[k],parametros[k],j) + epsilon*H;
            Xtotal[iteraciones++] = Xn[k];
              
            newH^=Xn[k];      
        }
        H = newH;
        /*Ahora, elegimos un valor para epsilon entre 1 y 8 bits.*/
        epsilon = (H % 3) - 1;
     
 

   } while (iteraciones < IT);
   
   /*Escribimos la informacion.*/
   fwrite(Xtotal,4,80000,archivobin); 

   if(!fclose(archivobin)){
      printf( "\nArchivo binario cerrado\n" );
   }
   else{
      printf( "\nError: Archivo binario no cerrado \n" );
      return 1;
   }
   
   

return 0;
}
\end{verbatim}

\subsection{Criterio 3.}
\begin{verbatim}
#include <stdio.h>
#include <stdlib.h>
#include <math.h>


#define RENYI_MAP(var, parametro, j) ((var)*(parametro)+((var)>>(j)))

#define  MAX 4294967295 /*El valor de 2^32-1.*/
#define TAMANIOCICLO 4294967296 /*EL valor de 2^32.*/
#define noMapas 4 /*NUmero de mapas a utilizar.*/
#define ITtotales 80000 /*Iteraciones totales para NIST.*/



/*
 File:   main.c
 Author: daniel
 El siguiente programa es un ejemplo de cuatro mapas caoticos RENYI acoplados.
 Para declarar nuestras variables, utilizamos los siguientes componentes:
 -Un arreglo que guarda el valor de los parametros para cada mapa.
 -Un arreglo que guarda el valor de los valores calculados para cada mapa.
*/


int main(){
 
   
   /*Declaramos los arreglos que vamos a utilizar para guardar esto.*/
   unsigned long Xn[noMapas];
   unsigned long Xtotal[ITtotales]; 
   unsigned long parametros[noMapas];
   unsigned int i;
   unsigned int epsilon;
   unsigned long gamma;
   unsigned long gammaComp;
   FILE *  archivobin; 

   unsigned long H=0; 
   unsigned int j=9; /*No mas de 16.*/
   unsigned long iteraciones=0;
   unsigned long IT = 80000;

  /* Apertura del fichero de destino, para escritura en binario.*/
   archivobin = fopen ("binarioXORcomp.dat", "wb");
   if (archivobin==NULL)
   {
   perror("No se puede abrir binarioXORcomp.dat");
   return -1;
   }
   

   /*Inicializamos nuestros parametros, en este punto se aplica el uso
    *de una llave, en este ejemplo todavia no elegimos una. Tambien,
    *los parametros son fijos en este ejemplo.*/
   parametros[0]=131071;
   Xn[0]=653;
   parametros[1]=104729;
   Xn[1]=769;
   parametros[2]=524287;
   Xn[2]=227;
   parametros[3]=65537;
   Xn[3]=823;
                   
   /*Primero, hacemos un ciclo inicial para calcular un nuevo valor para cada
   uno de los mapas y calcular, por primera vez, el resultado de la operacion
   XOR.*/
   for( i =0; i<noMapas; i++){
       Xn[i]= RENYI_MAP(Xn[i],parametros[i],j);
       H^=Xn[i];
   }
   
 
    /*Elegimos un epsilon aleatorio entre 1 y 16 (para operaciones de 32 bits).
      En este caso, elegimos uno que este entre 0 y 15.*/
   epsilon = 5;   
  
   gamma= pow(2,epsilon);
   gamma-=1;
   gammaComp= MAX-gamma;
    
   unsigned int k;
   unsigned long newH;

   do {
        
        newH = 0;
        for(k=0;k<noMapas; k++){            
            Xn[k]= gammaComp^RENYI_MAP(Xn[k],parametros[k],j)+ gamma^H;
            Xtotal[iteraciones++] = Xn[k];
     
            newH^=Xn[k];      
        }
        H = newH;
        /*Ahora, elegimos un valor para epsilon entre 1 y 8 bits.*/
        epsilon = (H % 8) + 1;  

   } while (iteraciones < IT);
 

   /*Escribimos la informacion.*/
   fwrite(Xtotal,4,80000,archivobin); 

   if(!fclose(archivobin)){
      printf( "\nArchivo binario cerrado\n" );
   }
   else{
      printf( "\nError: Archivo binario no cerrado \n" );
      return 1;
   }  
   

return 0;
}
\end{verbatim}


\end{document}

