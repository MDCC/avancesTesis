\chapter*{Capítulo 2}
\label{Fundamentos de Cifrado Basado en Caos}
\section{Caos}

Se dice que un mapa $f$ que está definido en el intervalo $I= [\alpha, beta]$ es caótico si se cumplen las siguientes condiciones:

\begin{enumerate}
\item Los puntos periódicos de $f$ son densos en I.
\item $f$ es una función transitiva en el intervalo $I$, esto significa que, dados dos subintervalos cualesquiera $U_{1}$ y $U_{2}$ en $I$, hay un punto $x_{0} \in U_{1}$y un $n>0$ tal que $f^{n}(x_{0}) \in U_{2}$.
\item $f$ tiene dependencia  a las condiciones iniciales en $I$; esto significa que hay una constante de sensibilidad $\beta$ tal que para cualquier $x_{o} \in I$ y cualquier intervalo abierto $U$ sobre $x_{0}$, hay alguna semilla $y_{o} \in U$ Y $n >0$ tal que

\begin{equation}
|f^{n}(x_{0}) - f^{n}(y_{0}) | > \beta.
\end{equation} 
\end{enumerate}