\chapter{Mapa caótico Renyi}


\section{Proceso de discretización}
El mapa caótico Renyi $\phi_{\beta} \colon [0,1) \longrightarrow [0,1)$ se define de la siguiente manera:
\begin{equation}
\phi (x)= (\beta \cdot x) \mod 1
\label{2a}
\end{equation}
donde $1 < \beta \in \mathbb{R}$; además, si $a$ y $b$ son números reales no negativos, se cumple lo siguiente: $b = a - \lfloor a/b \rfloor b$. La función suelo $\lfloor \cdot \rfloor$ regresa el mayor número entero menor o igual que el argumento.

De aquí en adelante, $\tilde{x}$ será una representación en punto fijo de $n$ bits donde 
\begin{equation}
\tilde{x}= \mathbb{Q}: \; \tilde{x} = \frac{k}{2^{n}}, \;\;\;\;\;\; k \in \mathbb{N}, \;\;\;\;\;\; k < 2^{n}.
\label{2b}
\end{equation}

De acuerdo a la ecuación~\ref{2b}, $\title{x}$ toma valores racionales en el intervalo $[0,1)$ y se tiene una cardinalidad de $2^{n}$.

Para lograr una evaluación de precisión finita de la ecuación~\ref{2a}, se utiliza una aproximación por truncamiento de la siguiente manera:
\begin{equation}
\tilde{\phi}_{\beta}(\tilde{x})= (\beta \cdot \tilde{x})_{tr} \mod 1 = \lfloor (2^{n} \cdot \beta \cdot \tilde{x}) \mod 2^{n}  \rfloor \cdot 2^{-n}.
\label{2c}
\end{equation}

La ecuación~\ref{2c} debe ser implementada en máquinas de estado finito; por lo tanto, $\beta$ puede ser, en general, cualquier número mayor que 1 cuya representación binaria requiera un número finito de dígitos. Por medio del isomorfismo $k = \tilde{x} \cdot 2^{n}$, el conjunto de posibles valores obtenidos en~\ref{2b} puede ser relacionado con el conjunto de números naturales
\begin{equation}
\Lambda_{n}= \{ k \in \mathbb{N}, \; 0 \leq k < 2^{n}  \}
\label{2d}
\end{equation}
y una función $f \colon  \Lambda_{n} \longrightarrow \Lambda_{n}$ puede ser definida como 
\begin{equation}
\begin{array}{lll}
f(k) & =2^{n}\tilde{\phi}_{\beta}(2^{-n}k)& = \lfloor  \beta \cdot  k \mod 2^{n} \rfloor \\
&   = \lfloor \beta \cdot k \rfloor \mod 2^{n}.  & 
\end{array}
\label{2e}
\end{equation}

La función $f$ es equivalente a~\ref{2c}. La ecuación~\ref{2e} representa al Generador de Números Pseudoaleatorios que se utiliza en el trabajo descrito en este documento. 

@@@@@@@@@@@@@@@@@@@@2Aqui voy jajaj




