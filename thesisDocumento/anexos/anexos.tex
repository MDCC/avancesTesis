\chapter*{Anexos}


\section*{Prueba de la Frecuencia}
La prueba consiste en medir la proporción de ceros y unos de la secuencia analizada. Se busca determinar si el número de ceros y unos en una secuencia es aproximadamente el mismo como debería ser para una secuencia verdaderamente aleatoria. La prueba evalúa que la fracción de unos  se acerce a $\frac{1}{2}$. Es importante recordar que todas las pruebas subsecuentes dependen del resultado de esta prueba.


\section*{Prueba de la Frecuencia por Bloques}
Esta prueba mide la proporción de unos en un bloque de $M$ bits, se busca determinar si la frecuencia de unos en el bloque es aproximadamente de $\frac{M}{2}$, que es el resultado esperado bajo la suposición de aleatoriedad. 

\section*{Prueba de las Corridas}
La proueba mide el número total de corridas en la secuencia, donde una corrida es una secuencia nininterrumpida de bits idénticos. Una corrida de longitud $k$ consiste de $k$ bits idénticos que está reada al inicio y al final con un bit del valor opuesto. Con esta prueba se busca demostrar si el número de coridad de ceros y unos de diversas longitudes es el que se espera para una secuencia aleatoria.

\section*{Prueba de la Corrida más Larga en un Bloque}

EL propósito de esta prueba es determinar si la longitud de la corrida más grande de unos en la secuencia probada es consistente con la longitud de la corrida más grande de unos que debería ser esperada en una secuencia aleatoria.




\section*{Prueba Binary Matrix Rank Test}
Esta prueba consiste en revisar la dependencia lineal entre subcadenas de un tamaño fijo de la secuencia original, esto se logra mediante el siguiente procedimiento: se construyen matrices de fragmentos sucesivos de la secuencia generada, y se revida la dependencia lineal de las filas o columnas de las matrices creadas. La desviación del rango, o deficiencia del rango, de las matrices de un valor esperado teóricamente es lo que genera el estadístico de interés.

\section*{Prueba (espectral) de la Transformada Discreta de Fourier}
Esta prueba está basada en la transformada Discreta de Fourier. Se detectan posibles caracterísitcas de la serie de bits que deberían indicar una desviación de la suposición de aleatoriedad.

Sea $x_{k}$ el $k-$ésimo bit, donde $k = 0, ... ,n-1$. Supongamos una codificación para los bits de $-1$ para el bit con valor cero y de $1$ para el bit uno. Utilizamos la siguiente expresión:

\begin{equation}
f_{j}= \sum_{k=0}^{n-1}x_{k} \exp \left(  2 \pi k j/n    \right)
\end{equation}

donde $\exp (  2 \pi kj/n ) = \cos  (  2 \pi  kj/n )  + \imath \sin (  2 \pi kj/n )  $, $j = 0, ... ,n-1$, además $ i = \sqrt{-1}$. Debido a la simetría de los valores de la transformada, únicamente los valores de $0$ a $(n/2 -1)$ son considerados. Ahora tomamos el módulo del número complejo $f_{j}$ y lo denotaremos por $\mod{j}$; baso la suposición de aleatoriedad de la serie $x_{j}$, se puede dar un intervalo de confianza para los valores de $\mod{j}$. Más específicamente, $95\%$ de los valores de $\mod{j}$ deberían ser menores que

\begin{equation}
h = \sqrt{\left( \log \frac{1}{0.05}n  \right)}.
\end{equation}

Un $p-valor$ basado en este umbral surge de la distribución binomial. Se $N_{l}$ el número de puntos más altos  menores que $h$, únicamente los primeros $\frac{n}{2}$ son considerados. Sea $N_{0} = .95N/2$ Y $d= (N_{1}- N_{0}) / \sqrt{n(0.95)(0.05)/4}$, el $p-valor$ es

\begin{equation}
2 \left( 1 - \phi \left(  \lvert d \rvert\ \right)  \right) = erfc \left( \frac{\lvert d \rvert\ }{ \sqrt{2}} \right)
\end{equation} 


\section*{Prueba de la Frecuencia}
\section*{Prueba de la Frecuencia}
