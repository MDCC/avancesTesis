\chapter*{Anexos}


\section*{Prueba de la Frecuencia}
La prueba consiste en medir la proporción de ceros y unos de la secuencia analizada. Se busca determinar si el número de ceros y unos en una secuencia es aproximadamente el mismo como debería ser para una secuencia verdaderamente aleatoria. La prueba evalúa que la fracción de unos  se acerce a $\frac{1}{2}$. Es importante recordar que todas las pruebas subsecuentes dependen del resultado de esta prueba.


\section*{Prueba de la Frecuencia por Bloques}
Esta prueba mide la proporción de unos en un bloque de $M$ bits, se busca determinar si la frecuencia de unos en el bloque es aproximadamente de $\frac{M}{2}$, que es el resultado esperado bajo la suposición de aleatoriedad. 

\section*{Prueba de las Corridas}
La proueba mide el número total de corridas en la secuencia, donde una corrida es una secuencia nininterrumpida de bits idénticos. Una corrida de longitud $k$ consiste de $k$ bits idénticos que está reada al inicio y al final con un bit del valor opuesto. Con esta prueba se busca demostrar si el número de coridad de ceros y unos de diversas longitudes es el que se espera para una secuencia aleatoria.

\section*{Prueba de la Corrida más Larga en un Bloque}

EL propósito de esta prueba es determinar si la longitud de la corrida más grande de unos en la secuencia probada es consistente con la longitud de la corrida más grande de unos que debería ser esperada en una secuencia aleatoria.




\section*{Prueba de la Frecuencia}
\section*{Prueba de la Frecuencia}
\section*{Prueba de la Frecuencia}
\section*{Prueba de la Frecuencia}
